\documentclass[12pt]{article}
\usepackage{lingmacros}
\usepackage{tree-dvips}
\usepackage[english]{babel}
\usepackage{amsthm}
\usepackage{hyperref}
\usepackage{amsmath}
\newtheorem*{claim*}{claim}
\newtheorem{theorem}{Theorem}[section]
\newtheorem{lemma}[theorem]{Lemma}
\newtheorem{claim}[theorem]{claim}
\newtheorem*{theorem*}{Theorem}
\newcommand{\ndiv}{\hspace{-4pt}\not|\hspace{2pt}}
\usepackage{amssymb}
\newtheorem{prop}{Proposition}

\begin{document}
\begin{center}
\section*{Basic Combinatorics - Spring, Home Assignment 3}
\subsubsection*{Saar Barak}
\end{center}
\subsection*{Problem 1.}
\begin{claim*}
There is an integer $n_0$ such that for any $n\ge n_0$, in every $9$-coloring of the integers $\{1,2,3,\dots,n\}$, one of the $9$ color classes contains $4$ integers $a,b,c,d$ such that $a+b+c=d$.
\end{claim*}
\begin{proof}
based on Ramsay Theorem Let $n_0=K(4,\dots,4)$, where 4 appears $k-1$ times. 
and lets $c$ be $r$-colouring s.t:\[c:\{1,\dots,n\}\to\{1,\dots, k\}\]
For graph $K_n$ and labelling of its edge $\{1,\dots,n\}$.
we can colour any edge $e_{ij}$ with $c(|i-j|)$. we got a $k-1$-colouring of $K_n$. then for $n_0$, we must have a $K_4$ with all  edges different. 
 for vertices $x\le y\le z \le w$ then \[a=y-x,b=z-y,c=w-z,d=w-x\] Gives a solution
\end{proof}
\subsection*{Problem 2.}
\begin{claim*}
every tournament on n vertices, contains a transitive tournament on
$\lfloor \log_2 n \rfloor$ vertices.
\end{claim*}
\begin{proof}Using induction for $n=0,1,2$ its holds on empty.
 W.L.O.G\footnote{we can modify any other tournament to to nearst power of 2 its will still hold for $\lfloor \log_2 n+1 \rfloor$ see (10)  } assume the claim holds for $n\leq 2^k$ now lets look at some tournament on $2^{k+1}$ vertices and we can pick any vertex $v$, and define:
 \[v_{in}=\{u : \text{ exsit edege } v\leftarrow u\}
 ,v_{out}=\{u : \text{ exsit edege } v\rightarrow u\}
 \]
Hence  $|v_{in}|+|v_{out}|=2^{k+1}-1$ and one of them contain $|2^k|$ edges, lets assume its  $v_{in}$\footnote{ its  equivalence for $v_{out}$} by our assumption its contains transitive tournament $T_{in}$  size $|k|$. 
now $T_{in}\cup \{v\}$ is sub tournament and any edge points to $v$ hence its transitive tournament on $|k+1|$ vertices.\end{proof}
\begin{claim*}
there exists a tournament on n vertices that does not contain a
transitive tournament on $2 \log_2 n + 2 $ vertices.
\end{claim*}
\begin{proof}
The number of Tournament on $n$ vertices is $2^{\binom{n}{2}}$.The number of tournaments of size $k$ is $k!$, and there are $\binom{n}{k}$ sets of size k, and the number of ways to choose the edges outside the transitive tournament is ${2^{\binom n2 - \binom k2}}$. hence if we show that 

\[k! \binom nk 2^{\binom n2 - \binom k2} < 2^{\binom n2}
\]
its yield that for some $k$   the number of n-vertex tournaments with a transitive subtournament on $k$ vertices is smaller than the total number of tournaments.
\item \begin{eqnarray}
2^{\binom n2} &>& k! \binom nk 2^{\binom n2 - \binom k2} \\
2^{\binom k2}&>& k! \binom nk  \\
&>& k!\frac{n!}{k!(n-k)!} \\
&>& n(n-1)(n-2) \dotsm (n-k+1)\\
&>& n^k 
\end{eqnarray}
Taking $\log_2$ from (1)(5)
\item \begin{eqnarray}
{\binom k2} &>& k\log_2(n)  \\
\frac{k!}{2(k-2)!}&>& k\log_2(n) \\
\frac{k!}{k(k-2)!}&>& 2\log_2(n)\\
k-1&>& 2\log_2(n)\\
k&>& 2\log_2(n)+1
\end{eqnarray}
\end{proof}
\subsection*{Problem 3.}
\begin{claim*}
if an n-vertex graph $G = (V, E)$ has no copy of $K_{2,t}$\footnote{I will use $t+1$ for the proof i.e $K_{2,t+1}$ } then
\[|E|\le {1\over 2}(\sqrt{t-1}n^{3\over 2}+n)
\]
\end{claim*}
\begin{proof}
W.L.O.G let $t\ge 1$. we can distinguish that any $e_1,e _2\in E$ have at most$^3$ $t$ neighbours. and each one of them can be part of pair.
we can consider it as the number of path length   2 in $G$.
Let $d(v_i)$ be the deg of $v_i\in G$ and we get that:
\item \begin{eqnarray}
t\binom n2\geq \sum_{v\in V} \binom{d(v)}{2}\geq n\binom {2|E|/n}{2}
\end{eqnarray}
The right-hand side hold from  Jensen’s Inequality and  since its minimized\footnote{convex property } the binomial when all the
degrees are equal, $d_i = 2|E|/|V|.$
\item \begin{eqnarray}
n\binom {2|E|/n}{2}= n\frac{(2|E|/n)(2|E|/n-1)}{2}\ge n\frac{(2|E|/n-1)^2}{2}
\end{eqnarray}
And
\item \begin{eqnarray} t\binom n2 = t\frac{n^2-n}{2}\le t\frac{n^2}{2}
\end{eqnarray}.We conclude from (11)(12)(13) that
\item \begin{eqnarray}
n\frac{(2|E|/n-1)^2}{2} &\leq & t\frac{n^2}{2} \\
{(2|E|/n-1)^2}&\leq & tn \\
2|E|/n&\leq & \sqrt{tn}+1 \\
|E|&\leq^{3} & {1\over 2}(\sqrt{t}n^{3\over 2}+n)
\end{eqnarray}
\end{proof}
\subsection*{Problem 4.}
\begin{claim*}
Let $S_1, \dots , S_n \in [n]$  such that $|S_i  \cap S_j | \le 1$ for all $1 \le i < j \le n$ then.
\[\frac{1}{n}\sum^n_{i=1}|S_i|=O(\sqrt{n})
\]
\end{claim*}
\begin{proof} Let define $G=(V,E)$ such that \[S=\{S_i:S_i\in [n]\},U=\{i\in n\}\text{ and }  E=\{e_{S_k,m}:m\in S_k \},   V =S\cup U  \]
Its immediate $|V|=2n$ and $G$ is Bipartite since we can dived $V$ into 2 disjoint  independent sets $S$ and $U$, that is  any $e\in E$ connects a vertex in $S$ to one in $U$. hence $G$ has no copy of $K_{2,2}$, using \textbf{Problem 3} we can get that 
\item \begin{eqnarray}
|E|&\le& {1\over 2}(\sqrt{2-1}(2n)^{3\over 2}+2n)\\
\sum^n_{i=1}|S_i|&\le& \sqrt{2}n^{3\over 2}+n\\
{1\over n} \sum^n_{i=1}|S_i|&\le& \sqrt{2}\sqrt{n}+2\\
\frac{1}{n}\sum^n_{i=1}|S_i|&=&O(\sqrt{n})
\end{eqnarray}
\end{proof}
\pagebreak
\subsection*{Problem 5.}
\begin{theorem*}
 if $G = (V, E)$ has no copy of $K_{t+1}$ then $|E|\leq (1-\frac{1}{t})\frac{n^2}{2}.$\\(Turan’s Theorem)
\end{theorem*}
\begin{proof}
Let $x = (x_1, . . . , x_n) \in \mathbb{R}^n$ and $f$ to be vector and weight function satisfying \[
\forall i\text{ } 0 <x_i\leq 1,\sum^n_{i=1} x_i = 1,f(x)=\sum_{i,j\in E} x_ix_j \]
By taking $x = (\frac{1}{n},\dots , \frac{1}{n})$ we get \item \begin{eqnarray}
f(x)\geq \sum_{i,j\in E}\frac{1}{n^2}\geq \frac{|E|}{n^2}\end{eqnarray}
The “weight shifting” method yield to shift the weight between any neighbours $x_i,x_j$ if $e_{i,j} \notin E$. \\We can notice that the sum is maximized when all the
weight is concentrated on a clique. Since any shift is does not decrease the value of $f$ we can repeat the processes. since $G = (V, E)$ has no copy of $K_{t+1}$ we can have at most $t$ size clique ,let name it $[T]$. we can get lower bound on $f$ :
\item \begin{eqnarray}
f(x)&\leq & \sum_{i,j\in [T]}x_ix_j=\sum_{i,j\in [T]}\frac{1}{t^2} \\
&\leq & \frac{t(t-1)}{2}\frac{1}{t^2}\\
&\leq & (1-\frac{1}{t})\frac{1}{2}
\end{eqnarray}
Combining (25) and (22) to finish the proof
\item \begin{eqnarray}
\frac{|E|}{n^2}&\leq & (1-\frac{1}{t})\frac{1}{2}\\
|E|&\leq & (1-\frac{1}{t})\frac{n^2}{2}
\end{eqnarray}

\end{proof}
\end{document}
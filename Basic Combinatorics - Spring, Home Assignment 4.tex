\documentclass[12pt]{article}
\usepackage{lingmacros}
\usepackage{tree-dvips}
\usepackage[english]{babel}
\usepackage{amsthm}
\usepackage{hyperref}
\usepackage{amsmath}
\newtheorem*{claim*}{claim}
\newtheorem{theorem}{Theorem}[section]
\newtheorem{lemma}[theorem]{Lemma}
\newtheorem{claim}[theorem]{claim}
\newtheorem*{theorem*}{Theorem}
\newcommand{\ndiv}{\hspace{-4pt}\not|\hspace{2pt}}
\usepackage{amssymb}
\newtheorem{prop}{Proposition}
\usepackage{mathrsfs} 
\newtheorem{Theorem}{Theorem}
\usepackage{babel}
\usepackage{calc}
\usepackage{graphicx}
\usepackage{caption}
\usepackage{lipsum}
\usepackage{hyperref}
\usepackage{amsmath}
\usepackage{dsfont}
\begin{document}
\begin{center}
\section*{Basic Combinatorics - Spring, Home Assignment 4}
\subsubsection*{Saar Barak}
\end{center}
\subsection*{Problem 1.}
\begin{prop}if $T(n) = T(n/3) + T(2n/3) + n$ then $T(n) = O(n \log n)$  \end{prop}
\begin{proof}
using induction, my induction hypothesis will  be
$$T(n) \leq C n \log n, \qquad \forall n < N,0<C  $$
Now we get that 
\begin{align}
T(n) &= T\left(\frac{n}{3}\right)+T\left(\frac{2n}{3}\right)+n \\
&\leq  C \frac{n}{3}\log \frac{n}{3} + C\frac{2n}{3}\log \frac{2n}{3} + n\\
&\leq C\frac{n}{3} \log n- C\frac{n}{3}\log 3 +C\frac{n}{3} \log n-  C\frac{2n}{3}\log \frac{3}{2} + n \\
&\leq Cn \log n- C\frac{n}{3}\log 3-  C\frac{2n}{3}\log \frac{3}{2} + n\\
&\leq^{(*)} Cn\log n
\end{align}
(2) imply the induction hypothesis, and (5) will hold by choosing $C$ s.t
\[ n \leq C\frac{n}{3}\log 3 +  C\frac{2n}{3}\log \frac{3}{2} 
\]
Hence for any  $n\geq 1$ we can choose $C$ such that
\[C \geq \frac{1}{\frac{1}{3}\log 3+\frac{2}{3}\log \frac{3}{2}} \thickapprox 1.578
\] We get that \[T(n) = O(n \log n)\]
\end{proof}
\pagebreak
\begin{prop}if $T(n) = 2T(n/2) + n \log n $ then $T(n) = O(n \log^2 n)$.\end{prop}
\begin{proof}
using induction, my induction hypothesis will  be
$$T(n) \le C n \log^2 n, \qquad \forall n < N,0<C$$
for $n=2$
\begin{align}
T(2) &= 2T\left(\frac{2}{2}\right)+2\log 2\\
&\le 2C+n\log n\\
&= O( n \log^2 n)
\end{align}
Hence for any  $n\geq 2$
\begin{align}
T(n) &= 2T(n)+n \log n\\
 &\le 2\log^2(\frac{n}{2})\frac{n}{2}+n \log n\\
 &= O(n \log^2 n)
\end{align}
(10) holds since $n>\frac{n+1}{2}$
\end{proof}

\begin{prop}Let $c_1, \dots , c_k$ be k positive reals satisfying $\sum^k_{i=1} c_i < 1$. if $T(n) =\sum^k_{i=1} T(c_in) +n
$ then $T(n) = O(n)$ .\end{prop}
\begin{proof}by induction, my induction hypothesis will  be
$$T(n)\le Cn, \qquad \forall n < N,0<C$$
Its immediate that $T(0)\le 0$ now lets
\begin{align}
T(n) &= \sum^k_{i=1} T(c_in) +n\\
 &\le\sum^k_{i=1} C(c_in) +n\\
 &\le Cn\left(\sum^k_{i=1}c_i +\frac{1}{C}\right)\\
 &\le^{(*)} Cn
\end{align}
(13) imply the induction hypothesis, and (15) will hold by choosing $C$ s.t
\[C \ge \frac{1}{1-\sum^k_{i=1}c_i}
\]
Hence for any $n\ge 1$ (15) holds and we get that \[T(n)=O(n)
\]
\end{proof}
\subsection*{Problem 2.}
\begin{theorem*}
 Every tournament has a Hamilton path
\end{theorem*}
 \begin{proof}[\proofname\ 1]Let $T=(V,E)$ be tournament graph where $|V|=n$ . using strong induction for $n\leq 2$ its immediate that  hamiltonian path exists. now lets assume its hold for any $k<n$. lets choose some  $v\in V$ and define 2 sets such that
\[V_{in}=\{u: \overrightarrow{(u,v)}\in E\},V_{out}=\{u: \overrightarrow{(v,u)}\in E\}
\]
Since $|V_{in}|<n,|V_{out}|<n$ by the induction hypothesis exists paths \\$P_{in}\in V_{in},P_{out}\in V_{in}$ such that $P_{in}$ and $P_{out}$ are hamiltonian paths. now the path $P_{in}\rightarrow v \rightarrow P_{out}$ is hamiltonian path for all vertices \\$|V_{in}|\cup |v| \cup|V_{out}|=n$
\end{proof}

 \begin{proof}[\proofname\ 2] First we can notice that $\chi(T)=$\footnote{T is graph on n vertex hence its can be colored by   at most $n$ different colours}$n$ since any 2 vertex connected with an edge . Since $\chi(T)\leq |P|$ where $P$ is the longest simple path in $T$. on the other hand its can be at most $n$ since $T$  have $n$ vertex .  we get that $|P|=n$.\\\\ Hence since $P$ is simple path i.e its visit any vertex of the $T$ exactly once, and $P$ visit all the vertex or in other worlds $P$ is Hamilton path in $T$
\end{proof}
\pagebreak
\subsection*{Problem 3.}
\begin{claim*}
 Any set X of $st + 1$ integers contains one of the following:
 \begin{list}{•}{}
 \item  A subset $T = \{ x_1,\dots , x_{t+1}\} \subseteq X$ of size $t + 1$ such that $x_k$ divides $x_{k+1}$ for every $1 \le k \le t$.
 \item A subset $S = \{ x_1,\dots , x_{s+1}\} \subseteq X$ of  $s + 1$ integers such that $x_i$ does not divides $x_j$ for every $x_i,x_j\in S$.
 \end{list}
\end{claim*}
 \begin{proof}
 Consider the following Poset define by \[\mathcal{P}=\footnote{by inculding $\langle 0,0 \rangle$ as well}\{X,\langle x_1,x_2 \rangle :x_1|x_2\}
 \]
Now lets look at $\mathcal{P}$ over $X$ , first notice that when its have chain size $|t+1|\Rightarrow$ exists sequence of $ x_1\preceq \dots \preceq x_{t+1}$ such that $x_1 | x_2 \dots |x_{t+1}\Rightarrow$ exist  $T\subseteq X$.
\\\ on the other hand if  $\mathcal{P}$ over $X$ ,  have anti-chain size $|s+1|\Rightarrow$ exist sequence of $ x_1\npreceq \dots \npreceq x_{s+1}$ such that $x_1 \nmid x_2 \nmid \dots  \nmid x_{s+1}\Rightarrow$ exist  $S\subseteq X$.
Since \[\omega(X)\alpha(X)\geq \omega(X) \frac{|X|}{\mathcal{X}(X)} \geq\footnote{Mirsky} |X|
\]
its following that splinting $X$ into  $\mathcal{X}(X)$ anti-chains, one of them will be at size $ \frac{|X|}{\mathcal{X}(X)}$.
if $\alpha(X)\geq s+1$ then $S\subseteq X$, else $\alpha(X)\leq s$ and
\[
\omega (X)\geq \frac{|X|}{\alpha(X)}\le \frac{st+1}{s} = t+\frac{1}{s}
\]
and we get that $\omega (X) \geq t+1\Rightarrow T \subseteq X$
 \end{proof}
 \subsection*{Problem 4.}
 Consider the following Poset define by \[\mathcal{P}=\{\mathscr{F},\langle S_1,S_2 \rangle :S_1\subseteq S_2\}
 \] where  $\mathscr{F}$ is collection $\mathscr{F} = \{ S_1, \dots , S_n\}$ of n sets.
 \pagebreak
 \begin{prop} both chain and anti-chain of $\mathcal{P}$  are union-free sets\end{prop}\begin{proof}
first by noticing that when its have chain $\Rightarrow$ exists sequence of $ S_1\preceq \dots \preceq S_{k}$ such that $S_1 \subseteq S_2 \dots \subseteq S_{k}$ let mark this set of element as $\mathcal{S}_{chain}$ .  lets assume that exsit some $S_i,S_j,S_k\in \mathcal{S}_{chain}$ s.t $S_i\cup S_j=S_k$,we can that hold only when $|S_i|,|S_j| \le |S_k|$ but the  Poset yield $S_i\preceq S_k$ and $S_j\preceq S_k$ hence $S_i=S_k$ or $S_j=S_k$ which lead to contradiction  since $\mathcal{S}_{chain}$ is set.\\\\
Define $\mathcal{S}_{anti-chain}$ such that   \[ \mathcal{S}_{anti-chain}=\{S_i,S_j : S_i\nsubseteq  S_j \wedge S_j\nsubseteq  S_i\quad \text{s.t } S_i,S_j \in \mathscr{F}\quad \forall i,j \} \]
 By assuming that exsit some $S_i,S_j,S_k\in \mathcal{S}_{anti-chain}$ s.t $S_i\cup S_j=S_k$. its followed that $S_i\subseteq S_k$ but $S_i\nsubseteq  S_j$ and we get an contradiction.
  \end{proof}

 \begin{claim*}
  every collection $\mathscr{F} = \{ S_1, \dots , S_n\}$ of n sets contains a sub-collection $S \subseteq  \mathscr{F}$ of at least
$\sqrt{n}$ sets which is union-free\end{claim*}
\begin{proof}
let $\alpha (\mathcal{P})$ be the longest $anti-chain$ of $\mathcal{P}$ over $\mathscr{F}$.  If $\alpha (\mathcal{P})\geq \sqrt{n}$ then exist such $S\in \mathscr{F}$
. and if $\alpha (\mathcal{P})< \sqrt{n}$
\[\frac{n}{\omega(\mathcal{P})} =\frac{|\mathscr{F}|}{\omega(\mathcal{P})}\leq \alpha (\mathcal{P})< \sqrt{n}
\]
\[\omega(\mathcal{P}) \geq \sqrt{n}
\]
And again by proposition 4 we get that exist such $S\in \mathscr{F}$

\end{proof}
\subsection*{Problem 5.}
\begin{claim*}
for  a finite poset $\mathcal{P}$ and let x, y be two elements of $\mathcal{P}$  that are incomparable under
$\mathcal{P}$ . then $\mathcal{P}$  has a linear extension in which $x < y$.
\end{claim*}
\begin{proof} Let $\mathcal{P}=(X,\preceq)$ be a finite partial order in which $x, y\in X$ are incomparable. now lets define new post $\hat{\mathcal{P}}=(X,\hat{\preceq})$  
\[
  \hat{\preceq} =
  \begin{cases}
   w\hat{\preceq}z &\text{if } w\preceq z  \\
   w\hat{\preceq}z &\text{if } z\preceq y\wedge x\preceq w \\
                           y \hat{\preceq} x        
  \end{cases}
\]
\begin{list}{•}{}
\item $\hat{\preceq}$ is reflexive since $\preceq$ is reflexive
\item $\hat{\preceq}$ is Transitive since $\preceq$ is Transitive , and we apply apply only steps that respect the Transitive property of   $\preceq$
\item $\hat{\preceq}$ is Anti-symmetric. consider $w\hat{\preceq}z$ and $z\hat{\preceq}w$  and let assume that $w\neq z$. if $x=w$ or $y=w$ or $x=w,y=z$ its immediate lead to  contradiction.\\ 
and when $w\hat{\preceq}z \Rightarrow z\preceq y\wedge x\preceq w$ and $z\hat{\preceq}w \Rightarrow w\preceq y\wedge x\preceq z$  then $x\preceq z \preceq y $ contradiction to the fact that $x,y$ are incomparable, hence $w\hat{\preceq}z$ and $z\hat{\preceq}w$ lead to $w=z$\footnote{i actually miss an case but its kind of similar proof}
\end{list}
Hence $\hat{\mathcal{P}}$ is poset where $x\hat{\preceq} y$ and its contains less incomparable pairs than $\preceq$ does. If $\hat{\mathcal{P}}$ is linear then we done. otherwise exists some incomparable  $w,z$ and we can extend $\hat{\preceq}$ to $\hat{\preceq}_1$ and follow the proses until we cover all the chains or get  some $\hat{\preceq}_k $ linear and respect $\mathcal{P}$ where $x<y$
\end{proof}
\subsection*{Problem 6.}
\begin{claim*}
in the setting of Arrow’s Theorem, if the individuals have only two options,
then they can come up with a non-dictator social choice function.
\end{claim*}
\begin{proof}
Lets proof that the democracy/majority voting system model satisfies the 3 condition of the Arrow’s Theorem when $N$ voters choose from $|\{A,B\}|=2$ choices.
\[F:\mathrm {S_2} ^{N}\to \frac{\sum^{N}_i \mathds{1}[S_i \text{ choose } A>B] }{N} \quad \text{i.e  and indicator if $S_i$ prefer }A \]
if $F\geq \frac{1}{2}$  return $(A,B)$ else $(B,A)$\\
\textbf{Monotonicity} 
For two preference profiles $R=(R_1, …, R_N)$ and $S=(S_1, …, S_N)$ such that both profiles prefer $A>B$ but
\[
0.5<
\frac{\sum^{N}_i \mathds{1}[S_i \text{ choose } A>B] }{N}<\frac{\sum^{N}_i \mathds{1}[R_i \text{ choose } A>B] }{N}
\]
more people support $(A,B)$ and its yield that $F$ is monotone\footnote{its the same idea for the monotone decrease case $B>A$ }\pagebreak

\textbf{Unanimity}
If alternative, $B<A$ for all orderings $R_1 , …, R_N,$ $R_i=(A,B)\forall i$ then $F(R_1, R_2, …, R_N)=(A,B) $ and $A$ is ranked strictly higher than $B$ by $F$. its immediate  from the way we construct $F$
\[0.5<\frac{\sum^{N}_i \mathds{1}[R_i \text{ choose } A>B] }{N}\]\\\\
\textbf{Non-dictatorship} There is no individual, i whose strict preferences always prevail consider the profile define
\[R=(R_1,R_2,\dots R_N)\quad \forall i R_i=(A,B)
\] 
And
\[S=(S_1,S_2,\dots S_N)\quad \forall i S_i=(B,A)
\]
For the 2 given profiles there is no individual who can change the result.
\end{proof} 
\subsubsection*{(*)}thanks for the tips!  just saw your notes on assignment 3 and I will apply them from now on $:)$
\end{document}